\documentclass{beamer}\usetheme{boxes}

\usepackage{amsmath,hyperref}

\newcommand{\Mp}{\ensuremath{{M_{+}}}}
\newcommand{\barMp}{\ensuremath{\overline{\Mp}}}
\newcommand{\Dt}{\ensuremath{\Delta t}}
\newcommand{\tM}[1]{\ensuremath{\tilde{M}(#1)}}
\newcommand{\tMt}{\tM{\tau}}
\newcommand{\tMB}{\ensuremath{{\tilde{M}_B}}}
\newcommand{\tE}[1]{\ensuremath{{\tilde{E}(#1)}}}
\newcommand{\tEt}{\tE{\tau}}

\newcommand{\bskip}{\\~\\}

\newcommand{\checkitem}{\item[\Checkmark]}
\newcommand{\xitem}{\item[\XSolidBrush]}
\newcommand{\mehitem}{\item[$\approx$]}
\newcommand{\easyframe}[2]{\frame{\frametitle{#1}
#2
}}

\setbeamertemplate{frametitle}[default][center]

\usepackage{Sweave}
\begin{document}
\input{template-concordance}
\title{Simulating Meth Production Networks}
\author{Carl~A.~B.~Pearson\inst{1} \and Burton~H.~Singer\inst{1} \and David~A.~Bright\inst{2}}
\institute{Emerging Pathogens Institute, University of Florida\inst{1} \and School of Social Sciences, University of New South Wales\inst{2}}
\date{19 FEB 14}
\frame{\titlepage}

\usebackgroundtemplate{
\vbox to \paperheight{\vfil\hbox to \paperwidth{\hfil\includegraphics[width=0.7\paperheight,height=0.7\paperheight]{2012army.png}\hfil}\vfil} }
\frame{
\frametitle{Supported by ARO Award \#W911NF-11-1-0036Z}
}
\usebackgroundtemplate{}

\easyframe{A SIMPLE LIST OF POINTS}{
\begin{itemize}
\item The Problem With Networks,
\item two,
\item see
\end{itemize}
% notes-notes-notes
}

\easyframe{NETWORKS ARE NOT\footnote{some physical systems aside} THE PHENOMENA}{
\begin{itemize}
\item The Problem With Networks,
\item two,
\item see
\end{itemize}

% notes-notes-notes
}

\easyframe{INSERTING AN R-GENERATED FIGURE}{
\begin{figure}
\begin{center}
\includegraphics{template-plotfig1}
\end{center}
\end{figure}
% notes-notes-notes
}

\frame{
\frametitle{INSERT ANOTHER PDF}
\begin{figure}
\begin{center}
\includegraphics[width=0.9\textwidth]{insert.pdf}
\caption{Bicout et al. J. Med. Entomol. 43(5): 936-946 (2006)}
\end{center}
\end{figure}
% this is a not uncommon measurement of a seasonally varying vector population.
}

\frame{
\frametitle{SHOW SOME MATH}
$$
M(t) = C\sin(\omega t+\theta)
$$
}

% plotfig1
\usebackgroundtemplate{\includegraphics[width=0.9\paperheight,height=0.9\paperheight]{template-plotfig1.pdf}}
\frame{
\frametitle{USE PREVIOUSLY GENERATING THING AS BACKGROUND}
\begin{itemize}
\item with
\item some
\item text
\item over it
\end{itemize}
}
\usebackgroundtemplate{}


\frame{
\frametitle{SEVERAL EQUATIONS}
\begin{align*}
E(t) &= \begin{cases}
        \dfrac{\Mp}{\Dt} & t \in \Dt \\
        0 & \textrm{otherwise}
        \end{cases}\tag{Step} \\
E(\rho, t) &= \begin{cases}
        \dfrac{2\Mp}{\Dt(2-\rho)} & t \in \Dt(1-\rho) \\
        \dfrac{2\Mp}{\Dt(2-\rho)\rho}\left(1-\dfrac{2|t|}{\Dt}\right) & t \in \rho\Dt \\
        0 & \textrm{otherwise}
        \end{cases}\tag{Modified Step} \\
E(t) &= \dfrac{2\Mp}{\Dt}\sqrt{\dfrac{2}{\pi}}e^{-\dfrac{8t^2}{\Dt^2}}\tag{Approximate $\delta$}
\end{align*}
}

% DEFINE LOTS OF R...

% THEN USE IT SEVERAL TIMES
\frame{
\frametitle{Modified Step (Trapezoid)}
\begin{figure}
\begin{center}
\includegraphics{template-pstrap}
\end{center}
\end{figure}
}

\frame{
\frametitle{Modified Step (Triangle)}
\begin{figure}
\begin{center}
\includegraphics{template-pstri}
\end{center}
\end{figure}
}

\frame{
\frametitle{Approx. $\delta$}
\begin{figure}
\begin{center}
\includegraphics{template-psdel}
\end{center}
\end{figure}
}

\frame{
\frametitle{USING COLUMNS EXAMPLE}
\begin{columns}[t]
\begin{column}{0.3\textwidth}
TEXT LEFT, FIG RIGHT\cite{someref}
\end{column}
\begin{column}{0.7\textwidth}
\begin{figure}
\begin{center}
\includegraphics[width=1.1\textwidth]{justone.pdf}
\end{center}
\end{figure}
\end{column}

\end{columns}
% notes-notes-notes
}

\frame{
\frametitle{BIBLIOGRAPHY EXAMPLE (CITE ON PREV SLIDE)}
\footnotesize
\bibliography{biblio}{}
\bibliographystyle{plain}
}

\end{document}
